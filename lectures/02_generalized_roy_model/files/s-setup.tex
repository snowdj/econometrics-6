\begin{frame}
	\begin{itemize}\setlength\itemsep{1em}
		\item Does the pursuit of comparative advantage increase or decrease earnings in equality within sectors and in the overall economy?
		\item Do the people with the highest $i$ skill actually work in sector $i$?
		\item As people enter a sector in response to an increase in the demand for its services, does the average skill level employed there rise or fall?
	\end{itemize}
\end{frame}
%-------------------------------------------------------------------------------
%-------------------------------------------------------------------------------
\begin{frame}
	\textbf{\cite{Roy.1951} Model}
	\begin{itemize}\setlength\itemsep{1em}
		\item Individuals are income maximizing, act under perfect information, and possess skills $S_1$ and $S_2$.
		\item The economy offers two employment opportunities associated with skill prices $\pi_1$ and $\pi_2$ and skill $i$ is only useful in sector $i$.
	\end{itemize}\medskip
	An individual chooses sector one if earnings are greater there:
	\begin{align*}
	w_1 > w_2 \quad\Longleftrightarrow\quad \pi_1 S_1 > \pi_2 S_2
	\end{align*}
\end{frame}

